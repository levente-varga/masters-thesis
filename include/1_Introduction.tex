\chapter{Introduction} \label{Chapter:Introduction}

% TODO: Look at other papers to see how they structure Introduction, Methods and Background

% Most similar paper: \cite{ljungdahl2020individual}

\section{Background}

In the second half of the twentieth century, often referred to as the dawn of the digital age or the prehistory of the computer game\cite{malliet2005history}, video games started to emerge as a new leading medium. Unlike today's blockbusters, which are often the result of the collaboration between multiple international studios, early video games were typically developed by just a handful of people and featured limited and basic gameplay elements and mechanics. 

Over time, genre boundaries began to blur, while entirely new genres started to form. At the same time, the gaming industry has undergone a rapid evolution in both production scale, quantity, and variety, and introduced new hybrid game genres through a process known as genre blending\cite{arsenault2009}, which combines key concepts from existing genres. This process has led to increasing commercial success, as genre-blended games usually stand out with their unique mix of game mechanics and diverse gameplay, appealing to a broader audience.



\subsection{A brief history of video games}

It is challenging to tell exactly when the video game era began, as there is ongoing debate over what constitutes the first video game. Some consider it to be Tennis for Two (1958)\cite{tennisfortwo1958}, an abstract 2D simulation of tennis played on an oscilloscope. However, its creator, Willy Higinbotham, developed it solely to demonstrate the capabilities of the hardware it was running on. As a result, there was no scoring system, nor any other feature that would resemble a modern video game\cite{malliet2005history}.

A stronger candidate for being the first true video game is Spacewar (1962) \cite{spacewar1962}, developed by Steve Russell during his time in college. Unlike Tennis for Two, Spacewar was intended to be a way of entertainment for two players. It was programmed on PDP-1 mainframe computers and featured a simplified representation of space with two controllable spaceships, a black hole located in the middle of the screen, and a few blinking stars for decoration. Players competed to destroy each other's spaceship using torpedoes while avoiding crashing into the black hole. Since Spacewar was a computer program intentionally developed for entertainment, it is widely regarded as the first true video game. Furthermore, it laid the groundwork for the action and simulation genres that emerged shortly after\cite{malliet2005history}.

Interestingly, most people would mistakenly believe that the first video game was Pong\cite{pong1972}, probably due to its massive popularity. However, it was released a full decade after Spacewar, in 1972.



\subsection{Genre combinations}

In 1993, Doom\cite{doom1993} was released, a prime example of a genre being born. Even though Doom was not the first of its kind, and just like other games, it was a combination of already existing game types like action and shooter, it became so successful and popular that the common term to describe similar games became "doom-clone". Only a couple of years later did a proper name "first-person shooter" (FPS) for these types of games appear and gain mass acceptance\cite{arsenault2009}, thus giving birth to a new video game genre.

Soulslike games share a similar story, although 'Soulslike' is not considered a genre of its own, but rather a subgenre. Its name comes from games developed by FromSoftware, a game development studio which is mainly famous for its games of high difficulty, namely Demon's Souls and the Dark Souls trilogy, all built around a very similar theme and game aesthetics, dynamics and mechanics\cite{hunicke2004mda}. Since these games are relatively new (the oldest of them, Demon's Souls, was released in 2009), a better and more descriptive name for this sub-genre has not been found yet.



\section{Thesis}



\subsection{Motivation}

Although numerous genre combinations already exist, there are still many unexplored possibilities that could lead to the creation of engaging and innovative games. By examining existing combinations and identifying gaps in the market, this research aims to propose new genre fusions that could offer new gameplay experiences. The ultimate goal is to develop a game based on one of these novel combinations.

Due to the limited time available for the thesis, the development of a complete game is unlikely. Instead, the feasibility of the proposed genre blend will be demonstrated by a game prototype that will retain all the characteristics and core concepts of each chosen genre, with limited visual appeal and polish. More about limitations will be discussed in section \ref{Section:Limitations}.



\subsection{Research questions}

The core motivation driving this research is to find out whether it is feasible to blend elements of a specific genre combination into a single, functional gameplay experience that has not been done before. The first part of this study is to discover these combinations and analyze their feasibility, then, as the second part, test it using a game prototype and gain feedback from playtesting sessions to examine the practical implications and player perceptions of such a hybrid game.

The following research questions will guide this research.

\textbf{RQ1:} \researchQuestionOne

The first part of the thesis will answer RQ1, first observing what game types exist already, then coming up with new ones, and finally, a genre combination will be chosen to be implemented into a game prototype to help answer RQ2, which is:

\textbf{RQ2:} \researchQuestionTwo

Since RQ1 serves as the base for the development of the game prototype, it must be answered earlier than RQ2. For that, chapter \ref{Chapter:Planning} was introduced. Thus, chapter \ref{Chapter:Implementation} will present only the implementation of the game prototype.  



\subsection{Goals and Challenges}

The main goal is to come up with a new video game genre combination and develop a functional game prototype that successfully combines all the elements and mechanics of the selected genres. Even if the resulting gaming experience turns out to be unenjoyable, it could still serve as a valuable lesson and a good indicator that these genres might not fit well together.

The core challenge - as with most games - is to achieve player satisfaction. Players who become frustrated and do not enjoy playing the game could indicate a potential design flaw and the wrong combination of genres. It could also mean that not the right players were selected for testing, therefore, it is crucial to conduct a brief 'pre-screening' of playtesters to ensure that the potential target audience are the ones testing the game. This is in an attempt to avoid unnecessary dissatisfaction, and thus unwanted outliers in the feedback data.



\subsection{Limitations} \label{Section:Limitations}

To allow more time for the development of the prototype, the research was affected by the following limitations:

\begin{itemize}
    \item An existing data set of games was used, which was not up-to-date. Getting the most recent data would have taken significant extra work while providing minimal impact on the result.
    \item Given the vast amount of possible combinations of video game genres, the complete list was not discussed. Instead, a few promising genres were pre-selected and tested to see if their combinations exist.
    \item Steam tags were used to find existing genre combinations. Although tags do not necessarily represent genres, Steam has an extensive database of its library, and tags are the most useful way of categorization found during research, making them a viable choice.
\end{itemize}

The game prototype was struck by the following limitations due to the limited amount of time:

\begin{itemize}
    \item The graphics, audio and other assets used were primitive to save time.
    \item The focus was on developing the core gameplay, instead of decorative features like a detailed settings page, achievements, or advanced accessibility features.
    \item The prototype was only developed for Windows, without any planned support for other platforms.
    \item The game only supports the 16:9 aspect ratio, since this is the most popular. Taking care of other aspect ratios would have been a waste of time regarding the goal of the thesis.
    \item There is no tutorial or any proper introduction to the game and its mechanics, since during playtests, players were given guidance when it seemed necessary.

    % TODO: Add more as the prototype progresses
    
\end{itemize}