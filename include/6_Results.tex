\chapter{Results} \label{Chapter:Results}

For describing the finished prototype.

\section{Playtests}

\subsection{Showcases}

During the early stages of development, I conducted gameplay presentations, where I sat down and showed live gameplay to others and asked for feedback and ideas. These sessions were especially helpful, guiding me in the right direction regarding design decisions in gameplay. These presentations showed the game in a super early state, so the goal was simply to get an idea of whether the project was heading in the right direction.

\subsection{Early in-house tests}

When the game reached a state in which it was considered playable having a fully functional gameplay loop, I started conducting in-house tests with friends. I let them play the existing part of the game without any prior explanations and watched how they explored the controls and tasks. From time to time I added comments on a certain feature missing when the player was approaching its intended place (i.e., extraction points were already present in the generated map but were not functional). The first of these tests were using a version of the game with limited card and enemy variation, horrible balancing, and the complete lack of meaningful meta-progression. Since the game was far from its intended state, the true goal of these tests was only to gain more early feedback about whether the core concept of the game is functional and can be built upon. Most of the reactions I received were about the lack of variety in the game, which was naturally expected. Suggestions were made to add more content and variety to the dungeon generation method, but in general, everyone thought that the combination of the extraction and deckbuilder genres was a unique and viable idea.

\subsection{Advanced in-house tests}

After responding to the feedback and finishing all the remaining tasks, a series of more serious tests was started, during which players were presented with a finished, feature-complete game. The test consisted of a complete playthrough of the game, as well as an extensive discussion afterwards, during which I took notes, guided by an early version of the feedback form used later in the public playtest. Three of these tests were conducted, with a gap of a few days between them that allowed the implementation of improvements and requested features.

During the first test, it became clear that the way the inventory and card upgrade menu was designed made it extremely slow and tedious to upgrade a large number of cards at once. In that version, the player had to first move all cards they were planning to upgrade from the inventory (or deck) to the stash by dragging them one by one from one container to the other. Furthermore, the cards were not sorted alphabetically or in any other way; their position in the list was determined by the order in which they were added to the container, so the player had to constantly search for the desired cards. At the workbench, they also needed to look through the whole list of cards placed in the stash to find the ones they wanted to upgrade, then drag them one by one to one of the upgrade slots, and then finally drag the upgraded version back to the stash container. Additionally, cards that could not be upgraded were also listed and could be dragged onto the upgrade area; only when all three upgrade slots were filled with cards would the game alert the player that the current card is already at its maximum level. Considering that most cards could be upgraded four times back then, to achieve the maximum upgrade for a single card, a total of 3 + 9 + 27 + 81 = 120 upgrade cycles were required, which is a ridiculous amount, especially if we take into account the time it takes to go through just a single cycle.

It was also pointed out that the game, especially the randomly generated maps that resulted from the procedural dungeon generation algorithm, felt boring and empty. As soon as the player had met all the enemies and found all the card types, there was nothing new or exciting to discover that would keep the player playing. The best way to solve the lack of this mystery factor would be a complete rework of the map generation method, with the introduction of biomes (distinct regions of the map) and secrets (rare rooms and enemies, areas locked behind special doors, etc.). However, all of these features would not fit the scope of this thesis.

The second test went much more smoothly, since most of the features requested from the first test were implemented by the time it was conducted, making the game much more enjoyable. All menus featured a sorted list of cards, and the cards could be dragged to the upgrade area from any container, as all of their contained cards were combined into a single list. The cards got an indicator in their top right corner showing which container they originate from, helping the player keep track of what cards they are using for the upgrade. However, the upgraded card variant was always put in the stash container, regardless of where its components originated. A good suggestion to solve this was made during the test: if at least one card originates from the deck, then the upgraded card should go to the deck instead. Similarly with the inventory: if at least one card originates from there and there are no cards from the deck, then the upgraded card goes to the inventory, otherwise the stash. This simple logic felt the most natural during testing, so it became the permanent solution. Additionally, even though the upgrade process has been dramatically sped up since the last test and the maximum level of cards was reduced to three, based on feedback, it still did not meet the expectations and was slow to use. However, making it even faster was no longer a high priority.

Since the first test, a new difficulty system has also been implemented that largely increased the feel of progression. Higher difficulty levels meant fewer extraction points, fewer bonfires, a higher number of enemies, and an even higher yield of loot from each of them.

Another requested feature was that the game should save occasionally. Since the game is a couple hours long, it would indeed be a good idea to implement saving to prevent loss of progression in case the tester cannot play the game for hours straight. However, saving should prevent players from escaping a deadly situation by simply closing and reopening the game and loading an earlier save. To accomplish this, the save file contained a flag about whether the last run started was formerly finished. When the game was launched, it checked for this flag, and if it was false, it treated the last run as if it was ended by player death, thus erasing all the unprotected cards from the deck and inventory.

Aside from additional balancing and detecting small bugs, the third and last test served as a confirmation that everything was working as intended and that the game could enter the public playtest state. 

\subsection{Public playtest}

After making sure that every part of the game was functional and thoroughly tested, I set up a project page on itch.io, created a feedback form using Google Forms, and advertised the game among university groups and friends. A deadline of one week counted from posting was set to conclude the test in time; however, this later was extended due to the surprisingly low number of downloads and form responses.

To make the test more accessible, testers were not required to complete a full playthrough of the game, as it would likely have taken multiple hours. Instead, they were asked to try to play the game as long as they could and then fill out the feedback form. During internal tests, the average time it takes to beat the game was estimated to be around two hours, but this result was achieved by continuously guiding the players when they felt stuck or did not understand a rule in the game.

Three days after the first demo version was released, I decided to implement an overhaul of the workbench menu that would make upgrading cards slightly faster. Instead of dragging cards from the list container over to the upgrade slots, they could simply click on a card to make it instantly jump to the first free slot. Similarly with the upgraded card: clicking on it puts it back into the list. This small enhancement made the upgrade process significantly faster, and thus likely reduced the total time it takes to beat the game, as it was shown during internal tests that card upgrades take up a considerable part of the gameplay loop.

Another correction was made in the 1.2 patch. Previously, the card sorting algorithm did not take into account whether a card was protected or not, and as a result, the order of protected and unprotected cards, which were otherwise identical, was undefined. This issue was caused due to a missing comparison between the protected flag of the card class. After the update, the protected cards were sorted ahead of others.