\chapter{Background} \label{Chapter:Background}

This chapter is about the literature review and the brief history of video games that will serve as a valuable background for future research.

\section{A brief history of video games}

It is difficult to tell what counts as the beginning of the video game era. There is a debate about what counts as the very first video game in history. Some consider it to be Tennis for Two (1958) \cite{tennisfortwo1958}, an abstract 2D simulation of tennis played on an oscilloscope. However, the creator, Willy Higinbotham, made Tennis for Two solely to demonstrate the capabilities of the machine it was running on. Therefore, there was no scoring system, nor any other feature that would resemble a modern video game. \cite{malliet2005history}

A better candidate for first place in the history of video games is Spacewar (1962) \cite{spacewar1962}, developed by Steve Russell while he was in college. In contrast with Tennis for Two, Spacewar was intended to be a way of entertainment for two players. It was programmed on PDP-1 mainframe computers and featured a simplified representation of space with two controllable spaceships, a black hole located in the middle of the screen, and a few blinking stars to decorate the screen. The task for the two players was to destroy the other's spaceship by firing torpedoes at it while avoiding crashing into the black hole. Since Spacewar was a computer program intentionally developed for entertainment, it is safe to say that it was the first true video game. Furthermore, it established the foundations of the action and simulation genres that emerged shortly after. \cite{malliet2005history}



\section{Defining video game genres}

The first attempts to classify video games occurred years after the first video game was released. In his book released in 1984, Chris Crawford listed games in two major categories, namely skill and action games (S\&A) which rely heavily on hand-eye coordination and motor skills and are usually fast-paced, and strategy games that emphasize more decision-making, and due to their nature, they generally take longer than S\&A games \cite{crawford1984art}. Within these main categories, Crawford defined multiple subdivisions for finer categorization. Some of these categories have evolved into modern game genres and are still present today, like sports and adventure games, while others, including paddle games, D\&D games and games of chance have completely merged into other genres and do not exist on their own. It is important to note that Crawford did not refer to these categories as genres.

Many of the first definitions of true video game genres were formulated by Mark J. P. Wolf, who also compared video games with the other existing media types, but mainly cinema \cite{wolf2002genre}. He concluded that video games and movies, while many similarities exist, can not be categorized the same way, due to their differences in interactivity (video games require active participation from the audience). Thus, the forty-one genres and subgenres explained in his paper \cite{wolf2002genre} are based on interactivity aspects, instead of iconography. The border between some of these genres is blurry, Racing and Driving, Catching and Capturing, Obstacle Course, and Platform all have a surprisingly similar definition to the other, differing only in a few parameters; however, these similarities are explicitly stated by the author.

Another approach was made in 2006 by Thomas H. Apperley, who agreed with Wolf that games cannot be categorized like other media types. He also criticized earlier game categorization attempts, mainly because they are "loose aesthetic clusters based around video games' aesthetic linkages to prior media forms" \cite{apperley2006genre}. Instead, his proposed framework tries to shape a new set of genres around the style of "ergodic interaction" that takes place within the games. In his paper, he defined four distinct video game genres. \textit{Simulation} games mimic the intricate processes of real-life systems, such as cars or cities. \textit{Strategy} games, including real-time strategy (RTS) and turn-based strategy (TBS) are outliers due to their similarity to board games. These games usually have a god's-eye view and information-heavy interfaces. \textit{Action} games include first-person shooters (FPS) and third-person games and focus more on the player's performance in combat manoeuvres and aiming with weapons, and unlike strategy games, they do not contain complex decision-making processes. And finally, \textit{role-playing} games (RPG), often compared to pencil-and-paper RPGs like Dungeons and Dragons, usually revolve around fantasy themes and focus on some form of character development, such as the collection of experience or skill points.

The early video game taxonomies were not applicable to the modern genres that followed \cite{starosta2024tangled}. Such complex genres as Massively Multiplayer Online Role-Playing Games (MMORPG) or Multiplayer Online Battle Arenas (MOBA) could not fit into the existing categories that were shaped, as they are a mix of several already defined game types (Multiplayer and Role-Playing).

Overall, it can be observed that the video game industry is and has always been struggling to come up with a universal organization system for games. Existing categorization methods, such as those used for books or movies, could not be applied due to differences between the mediums \cite{lee2014facet}.

Instead of sticking to one taxonomy, Steam \cite{steam}, an online video game store,  a list of user-defined tags (one or a few words) to categorize games published on the platform. Tags are similar, and sometimes identical, to the name of genres discussed in the above-mentioned papers, however, there is no definition behind them. %TODO: continue



\section{Genre combinations}

In 1993, Doom \cite{doom1993} was released, a prime example of a genre being born. Even though Doom was not the first of its kind, and just like other games, it was a combination of already existing genres like action and shooter, it became so successful and popular that the common term to describe similar games became "doom-clone". Only a couple of years later did a proper name "first-person shooter" (FPS) for these types of games appear and gain mass acceptance \cite{arsenault2009}, thus giving birth to a new video game genre.

Soulslike games share a similar story, although 'Soulslike' is not considered a genre of its own but rather a subgenre. Its name comes from games developed by FromSoftware, a game development studio that is mainly famous for its games of high difficulty, namely Demon's Souls and the Dark Souls trilogy, all built around a very similar theme and game aesthetics, dynamics and mechanics \cite{hunicke2004mda}. Since these games are relatively new (the oldest, Demon's Souls, was released in 2009), a better and more descriptive name for this sub-genre has not been found yet.



\section{Game engines}

This thesis will include the development of a game prototype, and to reduce development time, a game engine will be used. There are many freely available game engines on the market at the time this thesis is being written, so in order to pick one, a short comparison needs to be made.

But first of all, what is a game engine? To answer this question, it would be a good idea to look back at the very first game engine. Doom Engine was developed for Doom (1993) by id Software , 