\chapter{Introduction} \label{Chapter:Introduction}

% TODO: Look at other papers to see how they structure Introduction, Methods and Background

In the second half of the twentieth century, often referred to as the dawn of the digital age or the prehistory of the computer game\cite{malliet2005history}, video games started to emerge as a new leading medium. The first video games, compared to today's blockbusters sometimes developed by multiple studios around the world, were usually made by a handful of people and featured very limited and basic gameplay elements and game mechanics. 


Later, the genre boundaries began to blur, while entirely new genres started to form. At the same time, the gaming industry has undergone a rapid evolution in both production size, quantity, and variety, and introduced new hybrid game genres through a process called genre blending\cite{arsenault2009} by combining the key concepts of already existing genres. This process has led to more and more commercial success, as genre-blended games usually stand out with their unique mix of game mechanics and more diverse gameplay that appeal to more players.



\section{Motivation}

Although numerous combinations already exist, there are still many unexplored possibilities that could lead to engaging and innovative games. By examining existing genre combinations and identifying gaps in the market, this research aims to propose new genre fusions that could offer unique gameplay experiences. The ultimate goal is to develop a game based on one of these novel combinations.

Due to the limited time available for the thesis, the development of a complete game is unlikely. Instead, the feasibility of the proposed genre blend will be demonstrated by a game prototype that will retain all the characteristics and core concepts of each chosen genre, with limited visual appeal and polish.



\section{Research questions}

The core motivation driving this research is to find out whether it is feasible to blend elements of a specific genre combination into a single, functional gameplay experience that has not been done before. The first part of this study is to discover these combinations and analyze their feasibility, then, as the second part, test it using a game prototype and gain feedback from playtesting sessions to examine the practical implications and player perceptions of such a hybrid game.

The following research questions will guide this research.

\textbf{RQ1:} \researchQuestionOne

The first part of the thesis will answer RQ1, first observing what game types exist already, then coming up with new ones, and finally, a genre combination will be chosen to be implemented into a game prototype to help answer RQ2, which is:

% Explain that the Methods section will only work with the methodology of RQ2, 

\textbf{RQ2:} \researchQuestionTwo

Since RQ1 serves as the background to RQ2, it must be answered earlier as part of the literature review in chapter \ref{Chapter:Background} instead of chapter \ref{Chapter:Implementation}. Thus, chapter \ref{Chapter:Methods} will only discuss the methodology used for finding answers to RQ2, and chapter \ref{Chapter:Implementation} will only present the implementation the game prototype required for RQ2.  



\section{Goals and Challenges}

The main goal is to develop a functional game prototype that successfully combines all the selected genres. Even if the blending of these genres does not result in an enjoyable gaming experience, it could still serve as a valuable experience and a good indicator that these genres might not fit together so well.

The main challenge - as with most games - is to achieve player satisfaction. Players who become frustrated and do not enjoy playing the game could indicate a potential design flaw and the wrong combination of genres. It could also mean that not the right players were selected for testing, therefore it is crucial to conduct a brief 'pre-screening' of playtesters to ensure that mainly the potential target audience is testing the game, and to avoid unnecessary dissatisfaction and thus unwanted outliers in the feedback data.