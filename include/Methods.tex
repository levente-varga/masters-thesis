\chapter{Methods} \label{Chapter:Methods}

This chapter describes the core problem the and how this research plans to find answers to it.



\section{Research questions}

The core motivation driving this research is to find out whether it is feasible to blend elements of a specific genre combination into a single and functional gameplay experience that has not been done before. The first part of this study is to discover these combinations and analyze their feasibility, then as the second part, test it using a game prototype and gain feedback from playtesting sessions to examine the practical implications and player perceptions of such a hybrid game.

The following research questions will guide this research:

\textbf{RQ1:} Which genres of games can be extracted and combined to create a new game type?

The first part of the thesis will answer RQ1, and as a result, a genre combination will be ready to be implemented into a game prototype to help answer RQ2, which is:

\textbf{RQ2:} What implications can be derived by implementing a game based on a combination of specific game genres?



\section{Goals and Challenges}

The main goal is to develop a functional game prototype that successfully combines all the selected genres. Even if the blending of these genres does not result in an enjoyable gaming experience, it could still serve as a valuable experience and a good indicator that these genres might not fit together so well.

The main challenge - as with most games - is to achieve player satisfaction. Players becoming frustrated and not enjoying playing the game could indicate a potential design flaw and the wrong combination of genres. It could also mean that not the right players were selected for testing, therefore it is crucial to conduct a brief 'pre-screening' of playtesters to ensure that mainly the potential target audience is testing the game, and to avoid unnecessary dissatisfaction and thus unwanted outliers in the feedback data.



\section{Approach}

\textbf{Part 1:} To find answers to RQ1, the thesis will begin with a short research and experimentation where genre combinations - that currently do not exist - will be explored and analyzed whether they could function well together or not. As a result of this part, a genre combination will be chosen with which the thesis can continue to its second part.

\textbf{Part 2:} To answer RQ2, the project will involve the design and iterative implementation of a game prototype that combines all of the chosen game genres from Part 1 and their corresponding elements and game design patterns. 

During the project, between the iterative cycles, the game prototype will be occasionally tested on small player groups to gain instant feedback that could help the development stay on track.

At the end of the project, game testing sessions will be conducted that will include presenting the game prototype to others and letting them play with it for a short time while being guided and supervised. At the end of each session, using a feedback form and verbal questions \cite{björk2015Interviews}, data will be collected that will show how the combined gameplay design patterns work together and what the players think of their experience.

Additionally, the game will be internally evaluated using the extended version of the GameFlow model\cite{sweetser2017gameflow}.

Finally, a data analysis will be carried out on the feedback collected during the tests. This involves data collected from surveys, but also statistical data from the game itself that can be visualized using various methods. \cite{björk2015DataVisualization} 

Furthermore, during the thesis, regular meetings will be conducted with the project supervisor Natasha Mangan to discuss progress and deal with any problems at hand.



\section{Time plan}

The table below contains all the planned deadlines. These time periods are longer than the course schedule would suggest, as this is a 30-credit master's thesis project. However, to allow for more flexibility, the project is started earlier.

\begin{center}
    \begin{tabular}{ l l l l }
        \textbf{Task}                   & \textbf{Start} & \textbf{End} & \textbf{Length} \\
        Detailed literature review      & December 1     & January  31  & 9 weeks         \\
        Exploring genre combinations    & January 1      & January 31   & 4 weeks         \\
        Developing the game prototype   & January 15     & April 1      & 10 weeks        \\
        Development testing             & February 1     & April 1      & 8 weeks         \\
        Game testing                    & April 1        & April 15     & 2 weeks         \\
        Reacting to feedback            & April 15       & April 26     & 2 weeks         \\
        Finishing up the Final Report   & April 29       & May 31       & 5 weeks         \\
    \end{tabular}
\end{center}



\section{Ethical considerations}

Anonymous feedback data will be collected during playtests. All playtesters will be informed about this in advance. 

Participation in the playtests will be voluntary and the participants can opt out at any time if they do not want to continue the playtest.